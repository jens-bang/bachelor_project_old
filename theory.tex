\section{Theory}

\subsection{CPU vs. GPU}

\subsection{Lattice Bolzmann Model}

\subsection{CUDA}
CUDA is nVidia's way of introducing general purpose GPU programming to a wider audience. CUDA utilizes the parallel nature of the GPU, allowing a low end computer to process data in a different and some times more efficient way.

The CUDA API exposes a set of tags which allows a programmer to easily specify what methods, in otherwise ordinary c-code, should be executed on the GPU. The way CUDA makes it possible is by using a preprocessor which parses the source code and makes different files for the individual architectures. Thus the normal c-code goes to whichever compiler the user chooses, and the parallel section of the code goes to a compiler specifically designed to generate PTX-code. PTX-code is essentially assembly which is not specific to any GPU. The PTX code is then translated into assembly code specifically targeted at the GPU.

This effectively separates the tedium of writing assembly code directly aimed at a specific GPU-architecture.

\subsubsection{Multithreading}

\subsubsection{Parallel processing}
